\chapter{مرور ادبیات}

\section{مجموعه‌داده‌ها}


در سال‌های اخیر، مجموعه‌داده‌های متعددی برای پشتیبانی از توسعه مدل‌های یادگیری عمیق در حوزه تصویربرداری پزشکی، به‌ویژه برای تشخیص و طبقه‌بندی خونریزی درون‌جمجمه‌ای (ICH) ایجاد شده‌اند. این مجموعه‌داده‌ها از نظر منبع، اندازه، قالب و وظایف خاصی که هدف قرار می‌دهند، متفاوت هستند. در ادامه به بررسی برخی از مهم‌ترین مجموعه‌داده‌هایی که در این حوزه مورد استفاده قرار گرفته‌اند، می‌پردازیم.

\subsection{مجموعه‌داده‌ی انجمن پرتوشناسی آمریکای شمالی
 \lr{(RSNA)}}
مجموعه‌داده‌ی 
\lr{RSNA Intracranial Hemorrhage Detection}\cite{rsna_hemorrhage_detection_kaggle,rsna_kaggle}
 که برای چالش یادگیری ماشین سال ۲۰۱۹ انجمن پرتونگاری آمریکای شمالی جمع‌آوری شده است، یکی از منابع برجسته در زمینه طبقه‌بندی خونریزی درون‌جمجمه‌ای محسوب می‌شود. این مجموعه‌داده،از چند مرکز پرتونگاری جمع آوری شده است که سه مؤسسه دانشگاه استنفورد در ایالات متحده، دانشگاه فدرال سائو پائولو در برزیل و بیمارستان دانشگاه توماس جفرسون در ایالات متحده شامل آنها می‌باشد. این مجموعه شامل تصویر سی‌تی‌اسکن مغزی ۲۵۳۱۲ بیمار است که از این میان، ۸۸۸۹ بیمار دارای انواع مختلف خونریزی درون‌جمجمه‌ای هستند. تصاویر سی‌تی‌اسکن در این مجموعه‌داده به فرمت
 \lr{DICOM}
 \LTRfootnote{Digital Imaging and Communications in Medicine}
  ارائه شده‌اند که استانداردی برای تصویربرداری پزشکی است. این مجموعه‌داده به‌طور گسترده‌ای در طبقه‌بندی انواع خونریزی مورد استفاده قرار گرفته و به‌عنوان منبعی بنیادی برای آموزش و اعتبارسنجی مدل‌های یادگیری ماشین که به تشخیص و طبقه‌بندی خونریزی درون‌جمجمه‌ای تبدیل شده است.

\subsection{مجموعه‌داده‌ی \lr{MosMedData}}
مجموعه‌داده‌ی
 \lr{MosMed}\cite{medmos_khoruzhaya2024expanded}
  یک مجموعه تک‌مرکزی است که از مرکز تشخیص و فناوری‌های تله‌مدیسین مسکو در روسیه جمع‌آوری شده است. این مجموعه‌داده به‌طور خاص برای تسهیل توسعه سیستم‌های هوش مصنوعی (AI) به منظور تشخیص و طبقه‌بندی ICH طراحی شده است. این مجموعه شامل ۸۰۰ اسکن CT مغزی است که هر اسکن شامل سری‌های متعددی از تصاویر DICOM می‌باشد. این اسکن‌ها شامل موارد مثبت و منفی ICH هستند، هرچند تعداد دقیق اسکن‌های مثبت ICH به‌طور صریح بیان نشده است. دسترسی به این مجموعه‌داده در فرمت DICOM تضمین‌کننده‌ی سازگاری آن با طیف گسترده‌ای از ابزارها و سیستم‌های پردازش تصویر است. MosMedData نقش حیاتی در پیشبرد ابزارهای تشخیصی مبتنی بر هوش مصنوعی ایفا می‌کند، به‌ویژه در زمینه تشخیص انواع مختلف خونریزی‌های ناشی از تروما یا سکته.

\subsection{مجموعه‌داده‌ی CQ500}
مجموعه‌داده‌ی CQ500 نیز یک مجموعه چندمرکزی مهم است که شامل داده‌های جمع‌آوری شده از پنج مرکز مختلف در هند می‌باشد. این مجموعه شامل ۴۹۱ اسکن CT سر است که برای چندین حالت حاد از جمله ICH حاشیه‌گذاری شده‌اند. از این میان، تقریباً ۱۹۵ اسکن به‌عنوان حاوی خونریزی درون‌جمجمه‌ای شناسایی شده‌اند. این مجموعه‌داده به فرمت DICOM ارائه شده و از استفاده آن در طیف وسیعی از کاربردهای تصویربرداری پزشکی پشتیبانی می‌کند. مجموعه‌داده‌ی CQ500 به‌ویژه برای طبقه‌بندی ICH، خونریزی‌های ساب‌دورال و سایر ناهنجاری‌های حاد در اسکن‌های CT اهمیت دارد. این مجموعه نقش مهمی در آموزش مدل‌هایی ایفا کرده است که در محیط‌های اضطراری، جایی که تشخیص سریع و دقیق ICH حیاتی است، به کار می‌روند.

\subsection{مجموعه‌داده‌ی PHE}
مجموعه‌داده‌ی PHE، اگرچه به‌طور خاص برای ICH هدف‌گذاری نشده است، اما به دلیل تمرکز آن بر وظایف قطعه‌بندی و طبقه‌بندی مرتبط با سایر آسیب‌شناسی‌های مغزی مانند تومورهای مغزی قابل توجه است. این مجموعه‌داده‌ی چندمرکزی شامل تعداد قابل توجهی از اسکن‌های CT سر است، اگرچه تعداد دقیق آن ذکر نشده است. تصاویر به فرمت DICOM ارائه شده‌اند که آن‌ها را برای طیف وسیعی از وظایف تجزیه و تحلیل تصویر قابل دسترس می‌سازد. مجموعه‌داده‌ی PHE به عنوان منبعی ارزشمند برای توسعه مدل‌های یادگیری عمیق که هدف آن‌ها قطعه‌بندی و طبقه‌بندی است، به‌ویژه در تمایز بین آسیب‌شناسی‌های مختلف مغزی، خدمت می‌کند.

\subsection{مجموعه‌داده‌ی خونریزی درون‌جمجمه‌ای عراق}
مجموعه‌داده‌ی خونریزی درون‌جمجمه‌ای عراق پیش‌بینی می‌شود که به عنوان منبعی ارزشمند برای پژوهش‌های آتی در زمینه تشخیص و طبقه‌بندی ICH عمل کند. در حالی که جزئیات خاصی در مورد تعداد مراکز دخیل یا ترکیب دقیق مجموعه‌داده ارائه نشده است، انتظار می‌رود این مجموعه به‌طور قابل‌توجهی به توسعه مدل‌های هوش مصنوعی در حوزه ICH کمک کند. این مجموعه احتمالاً شامل مجموعه‌ای جامع از اسکن‌های CT خواهد بود که تمرکز آن بر ارائه داده‌های برچسب‌گذاری شده برای وظایف مختلف مرتبط با ICH است. فرمت تصاویر و جزئیات بیشتر برای ادغام آن در چارچوب‌های موجود برای تجزیه و تحلیل تصاویر پزشکی بسیار مهم خواهد بود.





\section{تحقیقات اخیر در زمینه یادگیری ماشین}

در سال‌های اخیر، استفاده از یادگیری عمیق در طبقه‌بندی و قطعه‌بندی خونریزی درون‌جمجمه‌ای پیشرفت‌های قابل‌توجهی را شاهد بوده است، به طوری که مطالعات زیادی در این زمینه در حال انجام است. توسعه و اعتبارسنجی این مدل‌ها نه تنها به دلیل نوآوری فنی، بلکه به دلیل پتانسیل ادغام آنها در سامانه تشخیص و درمان بیمارستان‌ها، که می‌تواند منجر به بهبود عملکرد کادر درمان، کاهش هزینه و زمان تشخیص و افزایش دقت در تشخیص خونریزی درون‌جمجمه‌ای شود، از اهمیت بالایی برخوردار است.

\lr{Chang} 
و همکاران یک مدل شبکه عصبی پیچشی
 \LTRfootnote{convolutional neural network}
  \lr{(CNN)}
 دوبعدی/سه‌بعدی را پیشنهاد کرده‌اند که هدف آن بهبود طبقه‌بندی و قطعه‌بندی خونریزی درون‌‌جمجمه‌ای در سی‌تی‌اسکن سر است. این روش از مزایای هر دو نوع
  \lr{CNN}
  ‎ سه‌بعدی و دو‌بعدی بهره می‌برد و نشان می‌دهد که ترکیب این دو تکنیک می‌تواند دقت تشخیص را بهبود بخشد. نتایج این مطالعه نشان می‌دهد که مدل پیشنهادی آن‌ها توانست با  
  \lr{Accuracy}
  برابر با
  \(82\%\)
  در تشخیص خونریزی عمل کند. با این حال، تمرکز این تحقیق بیشتر بر جنبه‌های فنی و معیارهای عملکرد مدل است و کمتر به بحث در مورد پیاده‌سازی آن در جریان‌های کاری بالینی پرداخته شده است\cite{chang2018hybrid}.

\lr{Chilamkurthy}
 و همکاران یک مدل یادگیری عمیق پیشنهاد کرده‌اند که برای شناسایی خونریزی درون‌جمجمه‌ای طراحی شده است. این مطالعه اهمیت تشخیص سریع و دقیق در محیط‌های فوریت پزشکی، جایی که تشخیص به موقع برای نتایج بیمار بسیار مهم است، را برجسته می‌کند. عملکرد مدل به‌طور دقیق با عملکرد پرتونگار‌ها مقایسه شده و پتانسیل استفاده از آن در محیط‌های بالینی نشان داده شده است. مدل پیشنهادی توانسته است با
 \lr{Accuracy}
 برابر با 
 \(90\%\)
 و 
 \lr{Sensitivity}
  برابر با 
 \(92\%\)
 عمل کند که نشان‌دهنده عملکرد قابل‌توجه آن در تشخیص خونریزی‌های درون‌جمجمه‌ای است‎\cite{chilamkurthy2018deep}.
 \lr{Titano}
  و همکاران یک سیستم یادگیری عمیق خودکار برای تشخیص خونریزی درون‌جمجمه‌ای معرفی می‌کنند که در سطح پرتونگار‌های خبره عمل می‌کند. این مطالعه نه تنها عملکرد سیستم را در مقایسه با کارشناسان انسانی تأیید می‌کند، بلکه پتانسیل ادغام آن در جریان‌های کاری بیمارستانی را نیز بررسی می‌کند. این سیستم در یک محیط بالینی آزمایش شد و نشان داد که می‌تواند در اولویت‌بندی بیماران کمک کند، به این ترتیب احتمال کاهش بار کاری پرتونگار‌ها و بهبود نتایج بیماران از طریق تشخیص سریع‌تر وجود دارد. این آزمایش در محیط بیمارستانی کاربرد عملی سیستم را در شرایط بالینی برجسته می‌کند. این سیستم توانسته است با 
 \lr{Accuracy}
 برابر با 
 \(87\%\)
 و 
 \lr{Sensitivity}
 برابر با 
 \(94\%\)
 عملکرد خود را نشان دهد‎\cite{titano2018automated}.

\lr{Kuo}
 و همکاران یک شبکه عصبی 
 \lr{CNN}
  را به‌طور خاص برای تشخیص خونریزی حاد درون‌جمجمه‌ای از سی‌تی‌اسکن سر توسعه دادند. این مدل وظایف طبقه‌بندی و قطعه‌بندی را با دقت‌هایی مشابه با پرتونگار‌های خبره دست می‌یابد. مدل پیشنهادی آن‌ها توانسته است با  
 \lr{Accuracy}
  برابر با 
 \(99\%\)
 و
 \lr{AUC}
 برابر با 
 \(0.991\)
 عمل کند، که نشان‌دهنده قابلیت اطمینان بالا در تشخیص است. 
 \lr{Kuo}
  و همکاران آزمایش‌هایی را در روند کاری بیمارستانی انجام دادند و عملکرد مدل را با پرتونگار‌ها در یک محیط بالینی مقایسه کردند. نتایج این آزمایش‌ها نشان می‌دهد که مدل می‌تواند به عنوان یک ابزار غربالگری مؤثر در بخش‌های فوریت‌های پزشکی به کار گرفته شود‎\cite{kuo2019expert}
.
\lr{Arbabshirani}
و همکاران استفاده از تکنیک‌های پیشرفته یادگیری ماشین را برای تشخیص خودکار خونریزی درون‌جمجمه‌ای مورد بررسی قرار می‌دهند و بر بهبود دقت مدل و کارایی محاسباتی تمرکز می‌کنند. مطالعه آنها به چالش‌های مربوط به پیاده‌سازی زمان واقعی
 \LTRfootnote{Real Time}
 در محیط‌های بالینی اشاره می‌کند. این تحقیق بینش‌های ارزشمندی در مورد پتانسیل یادگیری ماشین برای بهبود فرایند تشخیص ارائه می‌دهد. مدل پیشنهادی آن‌ها توانسته است با دقت 
 \lr{Accuracy}
  برابر با 
 \(86\%\)
 و حساسیت 
 \lr{Sensitivity}
  برابر با 
 \(89\%\)
 در تشخیص خونریزی‌های درون‌جمجمه‌ای عمل کند
\cite{arbabshirani2018advanced}.



\subsection*{مسائل موجود در ادبیات تحقیق و اهمیت پروژه}




با وجود پیشرفت‌های قابل‌توجه در مدل‌های یادگیری عمیق، هنوز چالش‌هایی مانند تفسیرپذیری محدود مدل‌ها، نیاز به داده‌های برچسب‌خورده گسترده، و دقت ناکافی در تشخیص برخی از زیرگونه‌های ICH وجود دارد. این چالش‌ها باعث شده‌اند که روش‌های فعلی برای استفاده در محیط‌های بالینی به صورت محدود مورد پذیرش قرار گیرند.

\subsection*{دستاوردهای ما و نحوه حل مشکلات گذشته}

در این پروژه، ما بهبودهایی بر روی مدل‌های موجود ایجاد کرده‌ایم که شامل بهبود تفسیرپذیری مدل‌ها با استفاده از تکنیک‌های جدید توجه (Attention)، افزایش دقت در تشخیص زیرگونه‌های مختلف ICH با استفاده از شبکه‌های عصبی سه‌بعدی و بهره‌گیری از روش‌های داده‌افزایی (Data Augmentation) برای کاهش نیاز به داده‌های برچسب‌خورده گسترده است. این بهبودها توانسته‌اند دقت و اعتمادپذیری تشخیص‌های خودکار را به میزان قابل‌توجهی افزایش دهند.
