\chapter{مرور ادبیات}

\section{مجموعه‌داده‌ها}


در سال‌های اخیر، مجموعه‌داده‌های متعددی برای پشتیبانی از توسعه مدل‌های یادگیری عمیق در حوزه تصویربرداری پزشکی، به‌ویژه برای طبقه‌بندی خونریزی درون‌جمجمه‌ای ایجاد شده‌اند. در ادامه به بررسی برخی از مهم‌ترین مجموعه‌داده‌هایی که در این حوزه مورد استفاده قرار گرفته‌اند، می‌پردازیم.

\subsection{مجموعه‌داده‌ی انجمن پرتوشناسی آمریکای شمالی
 \lr{(RSNA)}}
مجموعه‌داده‌ی 
\lr{RSNA Intracranial Hemorrhage Detection}\cite{rsna_hemorrhage_detection_kaggle,rsna_kaggle}
 که برای چالش یادگیری ماشین سال ۲۰۱۹ انجمن پرتونگاری آمریکای شمالی جمع‌آوری شده است، یکی از منابع برجسته در زمینه طبقه‌بندی خونریزی درون‌جمجمه‌ای محسوب می‌شود. این مجموعه‌داده،از چند مرکز پرتونگاری جمع آوری شده است که سه مؤسسه دانشگاه استنفورد در ایالات متحده، دانشگاه فدرال سائو پائولو در برزیل و بیمارستان دانشگاه توماس جفرسون در ایالات متحده شامل آنها می‌باشد. این مجموعه شامل تصویر سی‌تی‌اسکن مغزی ۲۵۳۱۲ بیمار است که از این میان، ۸۸۸۹ بیمار دارای انواع مختلف خونریزی درون‌جمجمه‌ای هستند. تصاویر سی‌تی‌اسکن درون این مجموعه‌داده درسطح برش،‌ حاشیه‌نویسی 
  \LTRfootnote{Annotation}
شده‌اند.  تصاویر سی‌تی‌اسکن در این مجموعه‌داده به فرمت
 \lr{DICOM}
 \LTRfootnote{Digital Imaging and Communications in Medicine}
  ارائه شده‌اند که استانداردی برای تصویربرداری پزشکی است. این مجموعه‌داده به‌طور گسترده‌ای در طبقه‌بندی انواع خونریزی مورد استفاده قرار گرفته و به‌عنوان منبعی بنیادی برای آموزش و اعتبارسنجی مدل‌های یادگیری ماشین که به طبقه‌بندی خونریزی درون‌جمجمه‌ای تبدیل شده است.

\subsection{مجموعه‌داده‌ی \lr{MosMed}}
مجموعه‌داده‌ی
 \lr{MosMed}\cite{medmos_khoruzhaya2024expanded}،
 یک مجموعه‌داده خونریزی درون‌جمجمه‌ای می‌باشد که در روسیه جمع‌آوری شده است. این مجموعه‌داده به‌طور خاص برای تسهیل توسعه سیستم‌های هوش مصنوعی به‌منظور تشخصی و طبقه‌بندی خونریزی درون‌جمجمه‌ای طراحی شده است. این مجموعه شامل سی‌تی‌اسکن مغزی ۸۰۰ بیمار است که 400 بیمار دارای خونریزی درون‌جمجمه‌ای هستند. این مجموعه داده درسطح بیمار حاشیه‌نویسی شده است و تصاویر آن به صورت فایل‌های 
 \lr{DICOM}
 دردسترس قرار دارد.

\subsection{مجموعه‌داده‌ی \lr{CQ500}}
مجموعه‌داده‌ی
 \lr{CQ500}\cite{cq500_chilamkurthy2018development}
 ، یک مجموعه‌داده مهم است که از چند مرکز متفاوت شامل پنج مرکز مختلف در هند می‌باشد. این مجموعه‌داده خاوی ۴۹۱ سی‌تی‌اسکن سر است که برای انواع خونریزی‌های درون‌جمجمه‌ای درسطح بیمار حاشیه‌گذاری شده‌اند. تصاویر سی‌تی‌اسکن در این مجموعه‌داده به صورت فایل
  \lr{DICOM}
   ارائه شده.
\subsection{مجموعه‌داده‌ی \lr{PHE-SICH-CT-IDS}}
مجموعه‌داده‌ی 
\lr{PHE-SICH-CT-IDS}\cite{PHE_ma2024phe}
، اگرچه به‌طور خاص برای خونریزی درون‌جمجمه‌ای جمع‌آوری نشده است، اما به دلیل تمرکز آن بر وظایف طبقه‌بندی، تشخیص و قطعه‌بندی مرتبط به 
\lr{Perihematomal Edema}
 در خونریزی‌ درون‌جمجمه‌ای قابل توجه است. این مجموعه‌داده‌ از بیمارستان 
 \lr{Shengjing}
 در چین جمع‌آوری شده است که 
  شامل تصویر سی‌تی‌اسکن 120 بیمار است که تمامی آنها خونریزی درون‌جمجمه‌ای دارند و حاشیه‌نویسی آنها در سطج برش انجام شده است. تصاویر سی‌تی‌اسکن در این مجموعه داده به صورت فایل‌های 
  \lr{NIFTI},
  \lr{JPG} و
  \lr{PNG}
  ارائه شده است. مجموعه‌داده‌ی 
  \lr{PHE-SICH-CT-IDS}
   منبعی ارزشمند برای توسعه مدل‌های یادگیری عمیق که هدف آن‌ها طبقه‌بندی، تشخیص یا قطعه‌بندی می‌باشد، است. در
  \autoref{fig:ch2-phe-sample}
چند برش از تصاویر مجموعه داده 
\lr{PHE-SICH-CT-IDS}
است، همانطور که در این تصویر مشخص است،‌ در اطراف ضایعه خونریزی،‌ یک حاشیه تیره‌تر وجود داره که به آن 
\lr{Edma}
گفته می‌شود و این ضایعه قطعه‌بندی شده است؛ همچنین این مجموعه‌داده حاشیه‌نویسی مناسب برای وظیفه تشخیص را نیز دارد.
\begin{figure}[H]
\centering
\includegraphics[width=1.0\linewidth]{"Images/Chapter2/PHE sample"}
\caption{چند نمونه تصویر از مجموعه‌داده 
\lr{PHE-SICH-CT-IDS}}
\label{fig:ch2-phe-sample}
\end{figure}

\subsection{مجموعه‌داده‌ی \lr{PhysioNet}}
مجموعه‌داده‌ی خونریزی درون‌جمجمه‌ای 
\lr{PhysioNet}\cite{physionet_hssayeni2020intracranial}،
مجموعه‌داده‌ای می‌باشد که در ادامه این مطالعه از آن استفاده شده است. این مجموعه داده
از بیمارستان
\lr{Al Hilla}
در عراق جمع‌آوری شده است و شامل 82 تصویر سی‌تی‌اسکن از بیماران است که 36 نفر از آنها دارای خونریزی درون جمجمه‌ای می‌باشند.
  این مجموعه داده، شامل حاشیه‌نویسی‌های مناسب برای وظایف طبقه‌بندی و قطعه‌بندی است که آن را به تنها مجموعه‌داده با دسترسی عمومی تبدیل می‌کند که امکان قطعه‌بندی خونریزی درون‌جمجمه‌ای را فراهم می‌کند. جزییات بیشتر درمورد این دیتاست در 
  \autoref{}
توضیح داده شده است.
تصویر
\autoref{fig:ch2-physionet-sample}
چند نمونه از برش‌های خونریزی درون این مجموعه‌داده را مشخص می‌کند.
\begin{figure}[H]
\centering
\includegraphics[width=1.0\linewidth]{"Images/Chapter2/physionet sample"}
\caption{چند نمونه تصویر از مجموعه داده
\lr{PhysioNet}}
\label{fig:ch2-physionet-sample}
\end{figure}


\section{تحقیقات اخیر در زمینه یادگیری ماشین}

در سال‌های اخیر، استفاده از یادگیری عمیق در طبقه‌بندی و قطعه‌بندی خونریزی درون‌جمجمه‌ای پیشرفت‌های قابل‌توجهی را شاهد بوده است، به طوری که مطالعات زیادی در این زمینه در حال انجام است. توسعه و اعتبارسنجی این مدل‌ها نه تنها به دلیل نوآوری فنی، بلکه به دلیل پتانسیل ادغام آنها در سامانه تشخیص و درمان بیمارستان‌ها، که می‌تواند منجر به بهبود عملکرد کادر درمان، کاهش هزینه و زمان تشخیص و افزایش دقت در طبقه‌بندی خونریزی درون‌جمجمه‌ای شود، از اهمیت بالایی برخوردار است.

\lr{Chang} 
و همکاران یک مدل شبکه عصبی پیچشی
 \LTRfootnote{convolutional neural network}
  \lr{(CNN)}
 دوبعدی/سه‌بعدی را پیشنهاد کرده‌اند که هدف آن بهبود طبقه‌بندی و قطعه‌بندی خونریزی درون‌‌جمجمه‌ای در سی‌تی‌اسکن سر است. این روش از مزایای هر دو نوع
  \lr{CNN}
  ‎ سه‌بعدی و دو‌بعدی بهره می‌برد و نشان می‌دهد که ترکیب این دو تکنیک می‌تواند دقت طبقه‌بندی را بهبود بخشد. نتایج این مطالعه نشان می‌دهد که مدل پیشنهادی آن‌ها توانست با  
  \lr{Accuracy}
  برابر با
  \(82\%\)
  در طبقه‌بندی خونریزی عمل کند. با این حال، تمرکز این تحقیق بیشتر بر جنبه‌های فنی و معیارهای عملکرد مدل است و کمتر به بحث در مورد پیاده‌سازی آن در جریان‌های کاری بالینی پرداخته شده است\cite{chang2018hybrid}.

\lr{Chilamkurthy}
 و همکاران یک مدل یادگیری عمیق پیشنهاد کرده‌اند که برای شناسایی خونریزی درون‌جمجمه‌ای طراحی شده است. این مطالعه اهمیت طبقه‌بندی سریع و دقیق در محیط‌های فوریت پزشکی، جایی که طبقه‌بندی به موقع برای نتایج بیمار بسیار مهم است، را برجسته می‌کند. عملکرد مدل به‌طور دقیق با عملکرد پرتونگار‌ها مقایسه شده و پتانسیل استفاده از آن در محیط‌های بالینی نشان داده شده است. مدل پیشنهادی توانسته است با
 \lr{Accuracy}
 برابر با 
 \(90\%\)
 و 
 \lr{Sensitivity}
  برابر با 
 \(92\%\)
 عمل کند که نشان‌دهنده عملکرد قابل‌توجه آن در طبقه‌بندی خونریزی‌های درون‌جمجمه‌ای است‎\cite{chilamkurthy2018deep}.
 \lr{Titano}
  و همکاران یک سیستم یادگیری عمیق خودکار برای طبقه‌بندی خونریزی درون‌جمجمه‌ای معرفی می‌کنند که در سطح پرتونگار‌های خبره عمل می‌کند. این مطالعه نه تنها عملکرد سیستم را در مقایسه با کارشناسان انسانی تأیید می‌کند، بلکه پتانسیل ادغام آن در جریان‌های کاری بیمارستانی را نیز بررسی می‌کند. این سیستم در یک محیط بالینی آزمایش شد و نشان داد که می‌تواند در اولویت‌بندی بیماران کمک کند، به این ترتیب احتمال کاهش بار کاری پرتونگار‌ها و بهبود نتایج بیماران از طریق طبقه‌بندی سریع‌تر وجود دارد. این آزمایش در محیط بیمارستانی کاربرد عملی سیستم را در شرایط بالینی برجسته می‌کند. این سیستم توانسته است با 
 \lr{Accuracy}
 برابر با 
 \(87\%\)
 و 
 \lr{Sensitivity}
 برابر با 
 \(94\%\)
 عملکرد خود را نشان دهد‎\cite{titano2018automated}.

\lr{Kuo}
 و همکاران یک شبکه عصبی 
 \lr{CNN}
  را به‌طور خاص برای طبقه‌بندی خونریزی حاد درون‌جمجمه‌ای از سی‌تی‌اسکن سر توسعه دادند. این مدل وظایف طبقه‌بندی و قطعه‌بندی را با دقت‌هایی مشابه با پرتونگار‌های خبره دست می‌یابد. مدل پیشنهادی آن‌ها توانسته است با  
 \lr{Accuracy}
  برابر با 
 \(99\%\)
 و
 \lr{AUC}
 برابر با 
 \(0.991\)
 عمل کند، که نشان‌دهنده قابلیت اطمینان بالا در طبقه‌بندی است. 
 \lr{Kuo}
  و همکاران آزمایش‌هایی را در روند کاری بیمارستانی انجام دادند و عملکرد مدل را با پرتونگار‌ها در یک محیط بالینی مقایسه کردند. نتایج این آزمایش‌ها نشان می‌دهد که مدل می‌تواند به عنوان یک ابزار غربالگری مؤثر در بخش‌های فوریت‌های پزشکی به کار گرفته شود‎\cite{kuo2019expert}
.
\lr{Arbabshirani}
و همکاران استفاده از تکنیک‌های پیشرفته یادگیری ماشین را برای طبقه‌بندی خودکار خونریزی درون‌جمجمه‌ای مورد بررسی قرار می‌دهند و بر بهبود دقت مدل و کارایی محاسباتی تمرکز می‌کنند. مطالعه آنها به چالش‌های مربوط به پیاده‌سازی زمان واقعی
 \LTRfootnote{Real Time}
 در محیط‌های بالینی اشاره می‌کند. این تحقیق بینش‌های ارزشمندی در مورد پتانسیل یادگیری ماشین برای بهبود فرایند تشخیص ارائه می‌دهد. مدل پیشنهادی آن‌ها توانسته است با دقت 
 \lr{Accuracy}
  برابر با 
 \(86\%\)
 و حساسیت 
 \lr{Sensitivity}
  برابر با 
 \(89\%\)
 در طبقه‌بندی خونریزی‌های درون‌جمجمه‌ای عمل کند
\cite{arbabshirani2018advanced}.



\subsection*{مسائل موجود در ادبیات تحقیق و اهمیت پروژه}




با وجود پیشرفت‌های قابل‌توجه در مدل‌های یادگیری عمیق، هنوز چالش‌هایی مانند تفسیرپذیری محدود مدل‌ها، نیاز به داده‌های برچسب‌خورده گسترده، و دقت ناکافی در تشخیص برخی از زیرگونه‌های ICH وجود دارد. این چالش‌ها باعث شده‌اند که روش‌های فعلی برای استفاده در محیط‌های بالینی به صورت محدود مورد پذیرش قرار گیرند.

\subsection*{دستاوردهای ما و نحوه حل مشکلات گذشته}

در این پروژه، ما بهبودهایی بر روی مدل‌های موجود ایجاد کرده‌ایم که شامل بهبود تفسیرپذیری مدل‌ها با استفاده از تکنیک‌های جدید توجه (Attention)، افزایش دقت در تشخیص زیرگونه‌های مختلف ICH با استفاده از شبکه‌های عصبی سه‌بعدی و بهره‌گیری از روش‌های داده‌افزایی (Data Augmentation) برای کاهش نیاز به داده‌های برچسب‌خورده گسترده است. این بهبودها توانسته‌اند دقت و اعتمادپذیری تشخیص‌های خودکار را به میزان قابل‌توجهی افزایش دهند.
