%کلمات کلیدی انگلیسی
\latinkeywords{Deep Neural Network, CT Scan Image Classification, CT Scan Image Segmentation, Intracranial Hemorrhage}
%چکیده انگلیسی

\en-abstract{
The rapid and accurate detection of intracranial hemorrhages using CT scan images has consistently been recognized as one of the most significant medical challenges in treating individuals with various brain injuries, strokes, and intracranial hemorrhages. The importance of this issue becomes apparent when even a few minutes of delay in diagnosis can lead to irreversible consequences for patients.
Given the complexity and high sensitivity of diagnosing such injuries, this process typically requires a high level of expertise and experience from physicians and radiologists. However, due to the limitations of human resources and the potential for human error, the need for the development of automated diagnostic systems based on deep learning has become increasingly evident.
In this context, the primary challenge for physicians, especially in emergency departments, is the accurate and rapid detection of hemorrhage regions in three-dimensional CT scan images. The performance of specialists in analyzing these images is influenced by their level of experience and environmental conditions.
The development of an intelligent assistant based on deep neural networks can improve medical processes in this area; however, developing such an assistant faces several challenges. Among these challenges are data imbalance, limited access to large datasets, and the variation in CT scan image quality across different imaging centers. These factors can reduce the accuracy of models in detecting hemorrhagic regions.
In this research, a two-stage method based on classification and segmentation, along with a post-processing step, has been developed using the PhysioNet intracranial hemorrhage dataset. In this research, the ResNet-50 model was used in the first stage and the U-Net model in the second stage, resulting in an IoU score of 0.22 and a Dice coefficient of 0.36. These results represent a significant improvement compared to the case where the two-stage method was not utilized.
}
%%%%%%%%%%%%%%%%%%%%% کدهای زیر را تغییر ندهید.

\newpage
\thispagestyle{empty}
\begin{latin}
\section*{\LARGE\centering Abstract}

\een-abstract

\vspace*{.5cm}
{\large\textbf{Key Words:}}\par
\vspace*{.5cm}
\elatinkeywords
\end{latin}