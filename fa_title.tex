%% -!TEX root = KNTUthesis.tex
% در این فایل، عنوان پایان‌نامه، مشخصات خود، متن تقدیمی‌، ستایش، سپاس‌گزاری و چکیده پایان‌نامه را به فارسی، وارد کنید.
% توجه داشته باشید که جدول حاوی مشخصات پروژه/پایان‌نامه/رساله و همچنین، مشخصات داخل آن، به طور خودکار، درج می‌شود.
%%%%%%%%%%%%%%%%%%%%%%%%%%%%%%%%%%%%
% دانشکده، آموزشکده و یا پژوهشکده  خود را وارد کنید
\faculty{دانشکده مهندسی برق}
% گرایش و گروه آموزشی خود را وارد کنید
\department{}
% عنوان پایان‌نامه را وارد کنید


\fatitle{پردازش تصاویر 
\lr{CT‌ Scan}
مغز به منظور قطعه‌بندی خونریزی داخلی مغز با استفاده از شبکه‌های عصبی عمیق
\\[.75 cm]
 پایان‌نامه کارشناسی}
% نام استاد(ان) راهنما را وارد کنید
\firstsupervisor{دکتر امیرحسین نیکوفرد}
%\secondsupervisor{استاد راهنمای دوم}
% نام استاد(دان) مشاور را وارد کنید. چنانچه استاد مشاور ندارید، دستور پایین را غیرفعال کنید.
%\firstadvisor{نام کامل استاد مشاور}
%\secondadvisor{استاد مشاور دوم}
% نام نویسنده را وارد کنید
\name{سید محمد }
% نام خانوادگی نویسنده را وارد کنید
\surname{حسینی}
%%%%%%%%%%%%%%%%%%%%%%%%%%%%%%%%%%
\thesisdate{شهریور 1403}

% چکیده پایان‌نامه را وارد کنید
\fa-abstract{
تشخیص سریع و دقیق خونریزی‌های درون‌جمجمه‌ای با استفاده از تصاویر سی‌تی‌اسکن، همواره به‌عنوان یکی از مهم‌ترین چالش‌های پزشکی در زمینه درمان افراد دارای انواع آسیب‌های مغزی، سکته‌های مغزی و خونریزی‌های درون‌جمجمه‌ای، مطرح شده است. اهمیت این موضوع زمانی آشکار می‌شود که حتی تأخیر چنددقیقه‌ای در تشخیص می‌تواند منجر به پیامدهای جبران‌ناپذیری برای بیماران شود.
باتوجه‌به پیچیدگی و حساسیت بالای تشخیص چنین آسیب‌هایی، این فرایند معمولاً نیازمند تخصص و تجربه‌ی بالای پزشکان و پرتوشناسان است. اما باتوجه‌به محدودیت منابع انسانی و احتمال خطاهای انسانی، نیاز به توسعه سامانه‌های خودکار تشخیص مبتنی بر یادگیری عمیق بیش‌ازپیش احساس می‌شود.
در این زمینه چالش اصلی برای پزشکان خصوصاً در بخش فوریت‌های پزشکی، تشخیص دقیق و سریع نواحی خونریزی در تصاویر سه‌بعدی سی‌تی‌اسکن است که عملکرد متخصصین در تحلیل این تصاویر، تحت‌تأثیر میزان تجربه آنها و شرایط محیطی قرار دارد.
توسعه یک دستیار هوشمند مبتنی بر شبکه عصبی عمیق، می‌تواند موجب بهبود فرایندهای پزشکی در این حوزه شود؛ اما توسعه این دستیار با چالش‌های متعددی روبرو است. از جمله این چالش‌ها می‌توان به عدم توازن داده‌ها، محدودیت در دسترسی به مجموعه‌داده‌های بزرگ، و تنوع کیفیت تصاویر سی‌تی‌اسکن در مراکز مختلف تصویربرداری اشاره کرد. این عوامل می‌توانند باعث کاهش دقت مدل‌ها در تشخیص نواحی دارای خونریزی شود. در این پایان‌نامه، با استفاده از مجموعه‌داده خونریزی درون‌جمجمه‌ای 
\lr{PhysioNet}،
 یک روش دومرحله‌ای مبتنی بر طبقه‌بندی و قطعه‌بندی، به همراه یک پس‌پردازش توسعه داده شده است. در این پژوهش با استفاده از مدل
\lr{ResNet 50}
در مرحله اول و مدل  
\lr{U-Net}
در محله دوم، معیار
\lr{IoU}
برابر با 
$0.22$
و معیار ضریب 
\lr{Dice}
برابر با 
$0.36$
کسب شده است که این نتایج نسبت به حالتی که از روش دومرحله‌ای استفاده نشده است،‌ بهبود چشمگیری داشته است.
 }


% کلمات کلیدی پایان‌نامه را وارد کنید
\keywords{شبکه عصبی عمیق، طبقه‌بندی تصاویر سی‌تی‌اسکن، قطعه‌بندی تصاویر سی‌تی‌اسکن، خونریزی درون‌جمجمه‌ای}



\KNTUtitle
%%%%%%%%%%%%%%%%%%%%%%%%%%%%%%%%%%
\vspace*{7cm}
\thispagestyle{empty}
\begin{center}
\includegraphics[height=5cm,width=12cm]{Images/besm.jpg}
\end{center}