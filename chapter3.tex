\chapter{نتایج}
\section{معیارهای ارزیابی در یادگیری عمیق}

در این بخش، به بررسی معیارهای مختلفی که برای ارزیابی مدل‌های یادگیری عمیق استفاده می‌شوند، می‌پردازیم. این معیارها شامل \lr{Sensitivity}، \lr{Specificity}، \lr{Precision}، \lr{F1 Score}، \lr{Accuracy}، \lr{Intersection over Union (IoU)} و ضریب \lr{Dice} می‌باشد.

قبل از پرداختن به تعریف این معیارها، ابتدا به تعریف چهار مفهوم اساسی می‌پردازیم که در فرمول‌های ارزیابی به‌کار می‌روند:

\begin{itemize}
    \item \textbf{\lr{TP} \lr{(True Positives)}}: تعداد نمونه‌های مثبت که به‌درستی به‌عنوان مثبت دسته‌بندی شده‌اند.
    \item \textbf{\lr{FP} \lr{(False Positives)}}: تعداد نمونه‌های منفی که به اشتباه به‌عنوان مثبت دسته‌بندی شده‌اند.
    \item \textbf{\lr{TN} \lr{(True Negatives)}}: تعداد نمونه‌های منفی که به‌درستی به‌عنوان منفی دسته‌بندی شده‌اند.
    \item \textbf{\lr{FN} \lr{(False Negatives)}}: تعداد نمونه‌های مثبت که به اشتباه به‌عنوان منفی دسته‌بندی شده‌اند.
\end{itemize}

\subsection{\lr{Sensitivity}}

\lr{Sensitivity}، که با نام نرخ تشخیص صحیح نیز شناخته می‌شود، معیاری برای ارزیابی توانایی مدل در تشخیص صحیح نمونه‌های مثبت است. فرمول این معیار در \autoref{eq:sensitivity} مشخص شده است.

\begin{latin}
\begin{equation}
\label{eq:sensitivity}
\text{Sensitivity} = \frac{TP}{TP + FN}
\end{equation}
\end{latin}

\subsection{\lr{Specificity}}

\lr{Specificity}، معیاری است که نشان‌دهنده توانایی مدل در شناسایی صحیح نمونه‌های منفی است. فرمول این معیار در \autoref{eq:specificity} مشخص شده است.

\begin{latin}
\begin{equation}
\label{eq:specificity}
\text{Specificity} = \frac{TN}{TN + FP}
\end{equation}
\end{latin}

\subsection{\lr{Precision}}

\lr{Precision}، معیاری برای ارزیابی میزان درستی دسته‌بندی نمونه‌های مثبت است. به عبارت دیگر، \lr{Precision} نسبت نمونه‌های مثبت درست دسته‌بندی شده به تمام نمونه‌های پیش‌بینی‌شده به عنوان مثبت است. فرمول آن در \autoref{eq:precision} مشخص شده است.

\begin{latin}
\begin{equation}
\label{eq:precision}
\text{Precision} = \frac{TP}{TP + FP}
\end{equation}
\end{latin}

\subsection{\lr{F1 Score}}

\lr{F1 Score}، میانگین موزون \lr{Precision} و \lr{Sensitivity} است که تعادلی بین این دو معیار ایجاد می‌کند. این امتیاز به‌ویژه در مواردی که تعادل بین \lr{Precision} و \lr{Sensitivity} اهمیت دارد، مورد استفاده قرار می‌گیرد. فرمول \lr{F1 Score} در \autoref{eq:f1_score} مشخص شده است.

\begin{latin}
\begin{equation}
\label{eq:f1_score}
\text{F1 Score} = 2 \times \frac{\text{Precision} \times \text{Sensitivity}}{\text{Precision} + \text{Sensitivity}}
\end{equation}
\end{latin}

\subsection{\lr{Accuracy}}

\lr{Accuracy}، نسبت نمونه‌هایی است که به‌درستی دسته‌بندی شده‌اند به تمام نمونه‌ها. این معیار نشان‌دهنده عملکرد کلی مدل است و فرمول آن در \autoref{eq:accuracy} مشخص شده است.

\begin{latin}
\begin{equation}
\label{eq:accuracy}
\text{Accuracy} = \frac{TP + TN}{TP + TN + FP + FN}
\end{equation}
\end{latin}

\subsection{\lr{Intersection over Union (IoU)}}

\lr{Intersection over Union}، که به عنوان \lr{Jaccard Index} نیز شناخته می‌شود، معیاری است که برای ارزیابی همپوشانی بین دو مجموعه، به‌ویژه در مسائل بخش‌بندی تصویر، استفاده می‌شود. فرمول \lr{IoU} در \autoref{eq:iou} مشخص شده است.

\begin{latin}
\begin{equation}
\label{eq:iou}
\text{IoU} = \frac{|A \cap B|}{|A \cup B|} = \frac{TP}{TP + FP + FN}
\end{equation}
\end{latin}

\subsection{\lr{Dice Coefficient}}

ضریب \lr{Dice}، مشابه \lr{IoU} است اما وزن بیشتری به ناحیه اشتراک می‌دهد و در مسائل بخش‌بندی تصویر بسیار مورد استفاده قرار می‌گیرد. فرمول ضریب \lr{Dice} در \autoref{eq:dice} مشخص شده است.

\begin{latin}
\begin{equation}
\label{eq:dice}
\text{Dice Coefficient} = \frac{2 \times |A \cap B|}{|A| + |B|} = \frac{2 \times TP}{2 \times TP + FP + FN}
\end{equation}
\end{latin}

\section{نتایج طبقه‌بندی}

در این بخش، نتایج حاصل از مدل‌های طبقه‌بندی خونریزی داخل‌جمجمه‌ای در تصاویر سی‌تی‌اسکن، بررسی و تحلیل شده‌اند. تمرکز اصلی آموزش مدر طبقه‌بندی بر روی بهینه‌سازی فراپارامترها برای مدل
 \lr{ResNet50}
  بوده است، که بهبودهای قابل توجهی در عملکرد مدل به‌ویژه در معیار \lr{Sensitivity}
   به همراه داشته است.

\subsection{نتایج برش‌محور }
پس از آموزش مدل طبقه‌بندی برای تمام 
\lr{fold}ها،
نمودار امتیاز
\lr{F1}
نسبت‌به آستانه‌های متفاوت رسم شد و در گام بعدی میانگین این نمودارها محاسبه شد. در انتها بهترین آستانه روی میانگین این نمودار‌ها محاسبه شده و در محاسبه معیارهای مربوط به زیرمجموعه ارزیابی استفاده شده است.
جدول \ref{tab:slice_level_results} نتایج به‌دست‌آمده از آموزش مدل \lr{ResNet50} و جزئیات هر یک از مراحل آموزشی را برای برش‌محور نشان می‌دهد. همانطور که در جدول مشاهده می‌شود، بهترین نتایج در مرحله 0 (\lr{Fold 0}) به‌دست آمده است که امتیاز \lr{F1} برابر با 0.62 را نشان می‌دهد. این نشان‌دهنده این است که توزیع داده‌ها در مجموعه‌های آموزشی و اعتبارسنجی این مرحله بیشتر شبیه به مجموعه آزمون است. در مقابل، مرحله 4 (\lr{Fold 4}) کمترین نتایج را به‌دست آورده، در حالی که نتایج مراحل دیگر با هم بیشتر هم‌خوانی دارند.

همچنین، همانطور که در شکل \ref{fig:f1_over_threshold} مشاهده می‌شود، با استفاده از تکنیک \lr{Voting}، امتیاز \lr{F1} بهبود قابل توجهی پیدا کرده است که نشان‌دهنده بهبود حساسیت مدل است. حساسیت مدل (\lr{Sensitivity}) در سطح برش 0.94 بوده که یک دستاورد قابل توجه در مقایسه با سایر مطالعات است.

\begin{table}[h!]
\centering
\caption{نتایج سطح برش مدل \lr{ResNet50}}
\label{tab:slice_level_results}
\begin{tabular}{|c|c|c|c|c|c|}
\hline
\textbf{مدل} & \textbf{حساسیت} & \textbf{اختصاصی} & \textbf{دقت} & \textbf{\lr{F1}} & \textbf{دقت کلی} \\ \hline
\lr{ResNet50 Fold 0} & 0.78 & 0.93 & 0.51 & 0.62 & 0.91 \\ \hline
\lr{ResNet50 Fold 1} & 0.68 & 0.90 & 0.38 & 0.49 & 0.88 \\ \hline
\lr{ResNet50 Fold 2} & 0.60 & 0.89 & 0.33 & 0.42 & 0.86 \\ \hline
\lr{ResNet50 Fold 3} & 0.86 & 0.89 & 0.41 & 0.55 & 0.89 \\ \hline
\lr{ResNet50 Fold 4} & 0.32 & 0.96 & 0.42 & 0.36 & 0.90 \\ \hline
\lr{Neethi et al. [19]} & 0.76 & - & 0.69 & 0.67 & 0.54 \\ \hline
\lr{ResNet50 Voting} & 0.94 & 0.91 & 0.49 & 0.64 & 0.91 \\ \hline
\end{tabular}
\end{table}

\subsection{نتایج سطح بیمار (\lr{Patient-Level})}

در سطح بیمار (\lr{Patient-Level})، طبقه‌بندی خونریزی بر اساس پیش‌بینی‌های سطح برش انجام شده است. یک بیمار در صورتی دارای \lr{ICH} در نظر گرفته می‌شود که حداقل یک برش از سی‌تی‌اسکن او به عنوان \lr{ICH} شناسایی شود. نتایج به‌دست‌آمده نشان می‌دهند که مدل \lr{ResNet50} در سطح بیمار با استفاده از تکنیک \lr{Voting}، به حساسیت 1.00، اختصاصی 0.80، دقت 0.75، امتیاز \lr{F1} برابر با 0.86 و دقت کلی 0.88 دست یافته است که از تمامی معیارهای گزارش شده در سایر مطالعات، برتری داشته است.

جدول \ref{tab:patient_level_results} نتایج سطح بیمار مدل \lr{ResNet50} را برای هر مرحله آموزشی نشان می‌دهد.

\begin{table}[h!]
\centering
\caption{نتایج سطح بیمار مدل \lr{ResNet50}}
\label{tab:patient_level_results}
\begin{tabular}{|c|c|c|c|c|c|}
\hline
\textbf{مدل} & \textbf{حساسیت} & \textbf{اختصاصی} & \textbf{دقت} & \textbf{F1} & \textbf{دقت کلی} \\ \hline
\lr{ResNet50 Fold 1} & 1.00 & 0.60 & 0.60 & 0.75 & 0.75 \\ \hline
\lr{ResNet50 Fold 2} & 1.00 & 0.60 & 0.60 & 0.75 & 0.75 \\ \hline
\lr{ResNet50 Fold 3} & 1.00 & 0.60 & 0.60 & 0.75 & 0.75 \\ \hline
\lr{ResNet50 Fold 4} & 0.83 & 0.80 & 0.71 & 0.77 & 0.81 \\ \hline
\lr{Kyung et al. [14]} & 0.97 & 0.74 & - & 0.84 & - \\ \hline
\lr{ResNet50 Voting} & 1.00 & 0.80 & 0.75 & 0.86 & 0.88 \\ \hline
\end{tabular}
\end{table}

\subsection{تحلیل بیشتر نتایج}

در این بخش، نتایج مرتبط با تفسیرپذیری مدل \lr{ResNet50} از طریق تکنیک‌های \lr{Grad-CAM} و \lr{t-SNE} تحلیل و بررسی می‌شوند.

\subsubsection{تحلیل \lr{Grad-CAM}}

تکنیک \lr{Grad-CAM} (\lr{Gradient-weighted Class Activation Mapping}) به عنوان یک ابزار قدرتمند برای تفسیر مدل‌های شبکه عصبی عمیق استفاده می‌شود. این تکنیک به ما اجازه می‌دهد تا ببینیم کدام بخش‌های تصویر ورودی بیشتر در تصمیم‌گیری مدل تأثیر داشته‌اند. در این پروژه، از \lr{Grad-CAM} برای ایجاد نقشه‌های حرارتی استفاده شده است که نواحی مهم تصاویر سی‌تی‌اسکن که منجر به تشخیص \lr{ICH} توسط مدل شده‌اند را برجسته می‌کند. شکل \ref{fig:grad_cam} نمونه‌ای از این نقشه‌های حرارتی را نشان می‌دهد که به وضوح مشخص می‌کند که مدل چگونه نواحی مختلف تصویر را برای تشخیص \lr{ICH} مورد ارزیابی قرار داده است.

\subsubsection{تحلیل \lr{t-SNE}}

تکنیک \lr{t-SNE} (\lr{t-Distributed Stochastic Neighbor Embedding}) یک روش برای کاهش ابعاد و تجسم داده‌های با ابعاد بالا است. در این پروژه، از \lr{t-SNE} برای تجسم توزیع ویژگی‌های استخراج‌شده از مدل \lr{ResNet50} استفاده شده است. شکل \ref{fig:tsne_representation} نشان‌دهنده توزیع ویژگی‌های تصاویر سی‌تی‌اسکن در فضای دو‌بعدی است که به‌وضوح نشان می‌دهد چگونه مدل قادر است تصاویر حاوی \lr{ICH} و غیر \lr{ICH} را از هم تفکیک کند. این تجسم کمک می‌کند تا دیدگاه بهتری نسبت به نحوه عملکرد مدل در سطح ویژگی‌ها داشته باشیم و نقاط ضعف و قوت آن را بهتر درک کنیم.

---

This extended LaTeX code not only presents the classification results but also includes a detailed analysis of the t-SNE and Grad-CAM techniques used in your research. You can integrate the provided tables and figures into your LaTeX document by replacing the placeholders with the correct paths or figure numbers.
