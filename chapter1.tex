\chapter{مقدمه}

\section{مقدمه}
\subsection{تشخیص خونریزی درون‌جمجمه‌ای}

خونریزی‌ درون‌جمجمه‌ای
 (\lr{ICH}\LTRfootnote{Intracranial Hemorrhage})
یک وضعیت اضطراری پزشکی است که تشخیص سریع و دقیق آن به‌منظور درمان مؤثر بیمار و کاهش خطر ناتوانی شدید یا مرگ، حیاتی است \cite{grewal2018radnet}.
\lr{ICH}
می‌تواند به دلایل مختلفی از جمله آسیب مغزی تروماتیک
\LTRfootnote{Traumatic Brain Injury}
، بیماری‌های عروقی، یا مشکلات مادرزادی ایجاد شود و بر اساس محل خونریزی در مغز طبقه‌بندی می‌شود \cite{monica2022detection}.
این طبقه‌بندی شامل خونریزی اپیدورال
\lr{(EDH)}\LTRfootnote{Epidural}
 ، خونریزی ساب‌دورال
 \lr{(SDH)}\LTRfootnote{Subdural}
  ، خونریزی ساب‌آراکنوئید
 \lr{(SAH)}\LTRfootnote{Subarachnoid}
   ، خونریزی پارانشیم مغزی 
 \lr{(CPH)}\LTRfootnote{Cerebral Parenchymal}
   ، و خونریزی داخل بطنی
 \lr{(IVH)}\LTRfootnote{Intraventricular }
  است \cite{burduja2020accurate,hssayeni2020intracranial}. 
 به‌صورت تقریبی سالانه بین 40000 تا 67000 بیمار دارای
\lr{ICH}
   در ایالات متحده آمریکا شناسایی می‌شوند که نرخ مرگ‌ومیر آنها در 30 روز اول حادثه در حدود 40 درصد است که در نتیجه آن، 
\lr{ICH}
   به یکی از بیماری‌ها با بیشترین آمار مرگ و میر تبدیل شده است. این در حالی است که عوارض دیگر این بیماری نیز بسیار خطرناک است، به‌عنوان‌مثال بیشتر از 46 درصد بیماران که خونریزی آنها از نوع 
 \lr{SAH}
 است، پس از بهبود به‌صورت دائمی دچار اختلالات شناختی می‌شوند
  ‎\cite{arbabshirani2018advanced,burduja2020accurate,morgenstern2010guidelines,van2010incidence,hackett2000health}‎.
  باتوجه‌به نرخ بالای مرگ‌ومیر مرتبط با 
  \lr{ICH}
  ، تشخیص سریع و دقیق 
  \lr{ICH}
  با استفاده از روش‌های تصویربرداری ضروری است \cite{kuo2019expert}. سی‌تی‌اسکن
 \LTRfootnote{Computed Tomography Scan}
   شایع‌ترین روش برای تشخیص سریع خونریزی خصوصا در مراکز فوریت‌های پزشکی به‌حساب می‌آید که دقت مناسب را برای تشخیص این بیماری به متخصصین می‌دهد \cite{ye2019precise,grewal2018radnet,arbabshirani2018advanced,chilamkurthy2018deep}.


\subsection{روش‌های مرسوم در تشخیص \lr{ICH}}

در حال حاضر، تصاویر سی‌تی‌اسکن (CT) بدون کنتراست، استاندارد طلایی برای تشخیص ICH هستند. در این روش، تصاویر دو بعدی از مغز به دست می‌آیند و متخصصین رادیولوژی با بررسی این تصاویر به صورت دستی، مناطق خونریزی را تشخیص می‌دهند. این فرآیند به دلیل وابستگی به تخصص و تجربه فردی، زمان‌بر و مستعد خطا است.تشخیص ICH به طور سنتی از طریق بازرسی بصری اسکن‌های CT توسط رادیولوژیست‌ها انجام می‌شود که به صورت دستی حضور و وسعت خونریزی را ارزیابی می‌کنند \cite{arbabshirani2018advanced}. این فرآیند زمان‌بر بوده و به شدت به در دسترس بودن رادیولوژیست‌های با تجربه بستگی دارد \cite{burduja2020accurate}. در شرایط اورژانسی، زمانی که برای تفسیر اسکن‌های CT صرف می‌شود می‌تواند به طور قابل توجهی بر نتایج بیماران تأثیر بگذارد، به ویژه در مواردی که مداخله فوری ضروری است \cite{chilamkurthy2018deep}. طبیعت دستی این فرآیند تشخیصی همچنین معرفی‌کننده احتمال خطای انسانی است، به ویژه در محیط‌های بالینی شلوغ که رادیولوژیست‌ها ممکن است تحت فشار زیاد باشند \cite{ye2019precise}. علاوه بر این، مطالعات نشان داده‌اند که اختلافات قابل توجهی بین تفسیرهای اولیه و نهایی اسکن‌های CT وجود دارد، که این مسئله ممکن است به اشتباهات در تشخیص برخی از انواع خونریزی مانند SDH و SAH منجر شود \cite{titano2018automated}.


\subsection*{مروری بر ادبیات تحقیق}

در سال‌های اخیر، استفاده از روش‌های یادگیری عمیق در تشخیص خودکار ICH مورد توجه بسیاری از پژوهشگران قرار گرفته است. مدل‌هایی مانند RADNet \cite{grewal2018radnet} و یک مدل سه‌بعدی مشترک از شبکه‌های عصبی پیچشی و بازگشتی \cite{ye2019precise} از جمله مدل‌هایی هستند که توانسته‌اند با دقت بالا و عملکرد قابل‌مقایسه با رادیولوژیست‌ها، ICH را تشخیص دهند.

\subsection*{مسائل موجود در ادبیات تحقیق و اهمیت پروژه}

با وجود پیشرفت‌های قابل‌توجه در مدل‌های یادگیری عمیق، هنوز چالش‌هایی مانند تفسیرپذیری محدود مدل‌ها، نیاز به داده‌های برچسب‌خورده گسترده، و دقت ناکافی در تشخیص برخی از زیرگونه‌های ICH وجود دارد. این چالش‌ها باعث شده‌اند که روش‌های فعلی برای استفاده در محیط‌های بالینی به صورت محدود مورد پذیرش قرار گیرند.

\subsection*{دستاوردهای ما و نحوه حل مشکلات گذشته}

در این پروژه، ما بهبودهایی بر روی مدل‌های موجود ایجاد کرده‌ایم که شامل بهبود تفسیرپذیری مدل‌ها با استفاده از تکنیک‌های جدید توجه (Attention)، افزایش دقت در تشخیص زیرگونه‌های مختلف ICH با استفاده از شبکه‌های عصبی سه‌بعدی و بهره‌گیری از روش‌های داده‌افزایی (Data Augmentation) برای کاهش نیاز به داده‌های برچسب‌خورده گسترده است. این بهبودها توانسته‌اند دقت و اعتمادپذیری تشخیص‌های خودکار را به میزان قابل‌توجهی افزایش دهند.
