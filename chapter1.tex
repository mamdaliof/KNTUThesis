\chapter{مقدمه}

\section{مقدمه}
\subsection{تشخیص خونریزی درون‌جمجمه‌ای}

خونریزی‌ درون‌جمجمه‌ای
\LTRfootnote{Intracranial Hemorrhage}
یک وضعیت اضطراری پزشکی است که تشخیص سریع و دقیق آن به‌منظور درمان مؤثر بیمار و کاهش خطر ناتوانی شدید یا مرگ، حیاتی است \cite{grewal2018radnet}.
خونریزی درون جمجمه ای
می‌تواند به دلایل مختلفی از جمله آسیب مغزی تروماتیک
\LTRfootnote{Traumatic Brain Injury}
، بیماری‌های عروقی، یا مشکلات مادرزادی ایجاد شود و بر اساس محل خونریزی در مغز طبقه‌بندی می‌شود \cite{monica2022detection}.
این طبقه‌بندی شامل خونریزی اپیدورال
\lr{(EDH)}\LTRfootnote{Epidural}
 ، خونریزی ساب‌دورال
 \lr{(SDH)}\LTRfootnote{Subdural}
  ، خونریزی ساب‌آراکنوئید
 \lr{(SAH)}\LTRfootnote{Subarachnoid}
   ، خونریزی پارانشیم مغزی 
 \lr{(CPH)}\LTRfootnote{Cerebral Parenchymal}
   ، و خونریزی داخل بطنی
 \lr{(IVH)}\LTRfootnote{Intraventricular }
  است \cite{burduja2020accurate,hssayeni2020intracranial}. 
 به‌صورت تقریبی سالانه بین 40000 تا 67000 بیمار دارای
خونریزی درون جمجمه ای
   در ایالات متحده آمریکا شناسایی می‌شوند که نرخ مرگ‌ومیر آنها در 30 روز اول حادثه در حدود 40 درصد است که در نتیجه آن، 
خونریزی درون جمجمه ای
   به یکی از بیماری‌ها با بیشترین آمار مرگ و میر تبدیل شده است. این در حالی است که عوارض دیگر این بیماری نیز بسیار خطرناک است، به‌عنوان‌مثال بیشتر از 46 درصد بیماران که خونریزی آنها از نوع 
 \lr{SAH}
 است، پس از بهبود به‌صورت دائمی دچار اختلالات شناختی می‌شوند
  ‎\cite{arbabshirani2018advanced,burduja2020accurate,morgenstern2010guidelines,van2010incidence,hackett2000health}‎.
  باتوجه‌به نرخ بالای مرگ‌ومیر مرتبط با 
  خونریزی درون جمجمه ای
  ، تشخیص سریع و دقیق 
  خونریزی درون جمجمه ای
  با استفاده از روش‌های تصویربرداری ضروری است \cite{kuo2019expert}. سی‌تی‌اسکن
 \LTRfootnote{Computed Tomography Scan}
   شایع‌ترین روش برای تشخیص سریع خونریزی خصوصا در مراکز فوریت‌های پزشکی به‌حساب می‌آید که دقت مناسب را برای تشخیص این بیماری به متخصصین می‌دهد \cite{ye2019precise,grewal2018radnet,arbabshirani2018advanced,chilamkurthy2018deep}.


\subsection{روش‌های مرسوم در تشخیص خونریزی درون‌جمجمه‌ای}

در حال حاضر تصاویر سی‌تی‌اسکن، به‌عنوان استاندارد اصلی و غیرتهاجمی
\LTRfootnote{Non-invasive}
برای تشخیص خونریزی‌ درون‌جمجمه‌ای است. سی‌تی‌اسکن یک نوع تصویر پرتونگاری
\LTRfootnote{Radiography}
سه‌بعدی است که متشکل از تصاویر دوبعدی از اندام بدن است. روش عمومی پردازش تصاویر سی‌تی‌اسکن به‌صورت دستی انجام می‌پذیرد که به‌موجب آن متخصصین پرتونگاری
\LTRfootnote{Radiology}
 و پزشکی، با بررسی برش‌های
\LTRfootnote{ُSlice}
سی‌تی‌اسکن را به‌صورت مجزا بررسی می‌کنند و مناطق خونریزی را تشخیص می‌دهند. این فرایند به دلیل وابستگی به تخصص و تجربه فردی، شرایط محیطی و فشار کاری، زمان‌بر و مستعد خطا است. \cite{arbabshirani2018advanced,grewal2018radnet,ye2019precise,chilamkurthy2018deep,kuo2019expert}.
فرایند بررسی دستی تصاویر سی‌تی‌اسکن، زمان‌بر بوده و به‌شدت به دردسترس‌بودن پرتونگار‌های 
\LTRfootnote{Radiologist}
باتجربه بستگی دارد \cite{burduja2020accurate}.
 در شرایط اضطراری، خصوصا در مراکز فوریت‌های پزشکی، زمانی که برای پردازش برش‌های سی‌تی‌اسکن صرف می‌شود، می‌تواند به طور قابل‌توجهی در نتایج درمان بیمارها تأثیر بگذارد؛ این مسئله در مواردی از اهمیت بیشتری برخوردار می‌شود که درمان بیمار نیازمند مداخله فوری گروه پزشکی است \cite{chilamkurthy2018deep}. نکته حائز اهمیت در روش معمول برای بررسی تصاویر سی‌تی‌اسکن در مراکز پزشکی این است که بررسی اولیه تصاویر، توسط پزشکان و پرتونگار‌هایی با تجربه کمتر انجام می‌شود و در مراحل بعدی این تصاویر توسط متخصصینی با تجربه بیشتر بررسی می‌شود. تعدادی از مطالعات نشان داده‌اند که در روش مذکور،‌ بین پزشکان و پرتونگار‌هایی که در مرحله اول تصاویر را بررسی می‌کنند و پزشکان و پرتونگار‌هایی که در ادامه این تصاویر را بررسی می‌کنند،‌ اختلاف‌نظر وجود دارد که این مسئله می‌تواند منجر به عواقب جبران‌ناپذیر گردد
\cite{ye2019precise, alfaro1995accuracy, lal2000clinical, erly2002radiology, strub2007overnight}.
   احتمال خطای انسانی در بررسی دستی تصاویر پیچیده و سه‌بعدی سی‌تی‌اسکن، از دیگر نقاط ضعف روش معمول پردازش این تصاویر است، به‌ویژه در محیط‌های شلوغ و پرتنش که پرتونگار‌ها ممکن است تحت فشار زیاد باشند \cite{ye2019precise}.
   
\section{روش‌های رایانه‌ای در پردازش تصاویر پزشکی}
\subsection{دستیارهای هوشمند مبتنی ‌بر شبکه‌های عصبی عمیق}
اهمیت مسئله خونریزی درون‌جمجمه‌ای و چالش‌های مرتبط با آن در بخش قبل مورد بررسی قرار گرفت، روش‌های مبتنی‌ بر پردازش رایانه‌ای
\LTRfootnote{Computer}
تصاویر پزشکی، می‌تواند یک راه‌حل مناسب برای رفع نقاط ضعف روش کنونی بررسی تصاویر پزشکی باشد. ابزارهای خودکار برای تشخیص و کمیت‌سنجی خونریزی، از پیشرفت‌های روش‌های یادگیری ماشین 
\LTRfootnote{Machine Learning}
و یادگیری عمیق
\LTRfootnote{Deep Learning}
 و سامانه‌های
 \LTRfootnote{System}
  تشخیص به کمک رایانه 
 \LTRfootnote{Computer-aided Diagnosis}
 استفاده می‌کنند تا تجزیه‌وتحلیل سریع و دقیقی از تصاویر سی‌تی‌اسکن ارائه دهند. با خودکارسازی تشخیص خونریزی درون‌جمجمه‌ای و استفاده از آنها به‌صورت نظر ثانویه
  \LTRfootnote{ُSecond Opinion}
 ، این سامانه‌ها می‌توانند بار کاری پرتونگار‌ها را کاهش دهند، دقت تشخیص را افزایش دهند از اشتباهات متخصصین جلوگیری کنند، زمان تشخیص را به حداقل برسانند، بعضی از هزینه‌های فرایند درمان را به علت کاهش دخالت انسانی کاهش دهند و به‌صورت کلی فرایند تشخیص را بهبود ببخشند که این موارد به بهبود نتایج بیماران منجر خواهد شد. بااین‌حال، ضمن اینکه سامانه‌های تشخیص به کمک رایانه نویدبخش هستند؛ اما امکان خطا در آنها وجود دارد که می‌تواند تصمیم‌گیری بالینی را با مشکلاتی روبرو کند؛ بنابراین، ادغام این ابزارها در عمل باید با دقت انجام شود \cite{titano2018automated}.

   
\subsection{مروری بر ادبیات تحقیق}

در سال‌های اخیر، استفاده از روش‌های یادگیری عمیق در تشخیص خودکار ICH مورد توجه بسیاری از پژوهشگران قرار گرفته است. مدل‌هایی مانند RADNet \cite{grewal2018radnet} و یک مدل سه‌بعدی مشترک از شبکه‌های عصبی پیچشی و بازگشتی \cite{ye2019precise} از جمله مدل‌هایی هستند که توانسته‌اند با دقت بالا و عملکرد قابل‌مقایسه با پرتونگار‌ها، ICH را تشخیص دهند.


\subsection*{مسائل موجود در ادبیات تحقیق و اهمیت پروژه}




با وجود پیشرفت‌های قابل‌توجه در مدل‌های یادگیری عمیق، هنوز چالش‌هایی مانند تفسیرپذیری محدود مدل‌ها، نیاز به داده‌های برچسب‌خورده گسترده، و دقت ناکافی در تشخیص برخی از زیرگونه‌های ICH وجود دارد. این چالش‌ها باعث شده‌اند که روش‌های فعلی برای استفاده در محیط‌های بالینی به صورت محدود مورد پذیرش قرار گیرند.

\subsection*{دستاوردهای ما و نحوه حل مشکلات گذشته}

در این پروژه، ما بهبودهایی بر روی مدل‌های موجود ایجاد کرده‌ایم که شامل بهبود تفسیرپذیری مدل‌ها با استفاده از تکنیک‌های جدید توجه (Attention)، افزایش دقت در تشخیص زیرگونه‌های مختلف ICH با استفاده از شبکه‌های عصبی سه‌بعدی و بهره‌گیری از روش‌های داده‌افزایی (Data Augmentation) برای کاهش نیاز به داده‌های برچسب‌خورده گسترده است. این بهبودها توانسته‌اند دقت و اعتمادپذیری تشخیص‌های خودکار را به میزان قابل‌توجهی افزایش دهند.
