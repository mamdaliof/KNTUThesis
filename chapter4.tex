\chapter{کارهای پیشنهادی و نتیجه‌گیری}
\section{محدودیت‌های و پیشنهادات}
خونریزی درون‌جمجمه‌ای یک وضعیت اضطراری پزشکی است که به دلایل متفاوت مثل آسیب‌های مغزی تروماتیک، بیماری‌های عروقی یا مشکلات مادرزادی ایجاد شود\cite{monica2022detection}.
جلوگیری از مرگ و کاهش عوارض ایجاد شده در خونریزی درون‌جمجمه‌ای نیازمند مداخله سریع است؛ اما به علت وجود عوامل محیطی مانند شلوغی مراکز درمان یا نبود متخصص مربوطه، تشخیص و درمان بیمار دارای خونریزی درون‌جمجمه‌ای می‌تواند با تأخیر انجام شود. علاوه‌بر این بررسی تصاویر پیچیده و سه‌بعدی سی‌تی‌اسکن نیز یکی دیگر از چالش‌ها در تشخیص دستی خونریزی درون‌جمجمه‌ای است.

این پژوهش در گام نخست با استفاده یک روش پردازش تصاویر سی‌تی‌اسکن مبتنی بر شبکه‌های عصبی عمیق،‌ یک دستیار هوشمند را برای استفاده در مراکز پزشکی توسعه داده و در گام بعدی با توسعه روش‌های موجود در پردازش تصاویر سی‌تی‌اسکن، دقت این دستیار را در زمینه طبقه‌بندی و قطعه‌بندی افزایش داده است.
استفاده از روش دومرحله‌ای موجب کاهش
\lr{Field-of-View}
در مدل قطعه‌بندی به 
\lr{Region-of-Interest}
می‌شود که در نتیجه آن، امکان ایجاد خطا در برش‌‌های سالم کاهش پیدا می‌کند. از دیگر روش‌های دومرحله‌ای به 
\lr{Cascaded U-Net}
و روش
\lr{Fine to coarse}
می‌توان اشاره کرد.
 نتایج به‌دست‌آمده از روش دومرحله‌ای پیشنهاد شده در این پژوهش به همراه پس‌پردازش استفاده شده،‌ نشان داد که عملکرد مدل می‌تواند بهبود قابل‌توجهی پیدا بکند که نشان‌دهنده این است که این روش می‌تواند در مراکز پزشکی استفاده شود.

بااین‌حال، چالش‌ها و محدودیت‌هایی در این تحقیق وجود دارد. یکی از مهم‌ترین چالش‌ها، دسترسی محدود به مجموعه‌داده‌های متنوع و بزرگ در حوزه طبقه‌بندی و قطعه‌بندی خونریزی درون‌جمجمه‌ای است. تنوع ناکافی در کیفیت تصاویر سی‌تی‌اسکن نیز ممکن است بر قابلیت تعمیم‌پذیری مدل‌های ما تأثیر بگذارد. همچنین، توان محاسباتی موردنیاز برای آموزش مدل‌ها و اجرای پس‌پردازش ممکن است باعث افزایش زمان پردازش شود که باید در آینده با استفاده از روش‌های بهینه‌سازی بهبود یابد.
یکی از محدودیت‌های اساسی در این پژوهش این است که برای پردازش یک تصویر سی‌تی‌اسکن،‌ لازم است تا این تصویر از 10 مدل شبکه عصبی عبور داده شود و سازوکار شورایی روی آنها اعمال شود که این مسئله توان محاسباتی موردنیاز را افزایش می‌دهد.

یکی از مهم‌ترین زمینه‌های موجود برای پژوهش‌های آینده، جمع‌آوری یکم مجموعه‌داده باکیفیت و حجم مناسب از مراکز موجود در ایران است که با استفاده از این مجموعه‌داده امکان ارزیابی عملکرد مدل در مراکز بهداشتی ایران فراهم شود. در ادامه استفاده از مجموعه‌داده‌های موجود مثل
\lr{RSNA}\cite{rsna_hemorrhage_detection_kaggle}
که تعداد بسیار زیادی تصویر مناسب برای طبقه‌بندی دارد و استفاده از روش‌های یادگیری انتقالی،‌ می‌تواند یکی از راه‌های بهبود عملکرد مدل‌های پردازش تصویر باشد. 
ازآنجایی‌که تصاویر سی‌تی‌اسکن یک توالی از برش‌ها است، اضافه‌کردن لایه‌های زمانی در مراحل تصمیم‌گیری می‌تواند در عملکرد مدل تأثیرگذار باشد. تصاویر سی‌تی‌اسکن ماهیت سه‌بعدی دارند؛ بنابراین توسعه مدل‌هایی که از لایه‌های پیچشی سه‌بعدی استفاده کنند یا روش‌های 
$2D+1D$
به‌منظور افزایش مشارکت برش‌ها در تصمیم‌گیری مدل یک زمینه تحقیقاتی است.
استفاده از مدل‌های جدید در زمینه پردازش تصاویر خصوصاً مدل‌های مبتنی بر مکانیزم 
\LTRfootnote{Mechanism}
توجه و مدل‌هایی که حساسیت آنها به شکل ضایعه بیشتر از بافت ضایعه باشد یکی از زمینه‌های موجود برای پژوهش‌های آینده است.
روش 
\lr{Model soups}
نیز می‌تواند یک زمینه جذاب برای توسعه عملکرد مدل‌ها برای استفاده در زمینه پزشکی باشد. 
توسعه سازوکار شورایی،‌ یک زمینه مناسب برای توسعه عملکرد مدل‌ها است،‌ با توجه به اینکه آستانه بهینه برای هر مدل می‌تواند با مدل دیگر متفاوت باشد و فاصله پیشبینی مدل از آستانه بهینه می‌تواند اطمینان پیشبینی را بالاببرد،‌ استفاده از روش‌هایی مبتنی بر ریاضیات فازی 
\LTRfootnote{Fazzy}
یک پیشنهاد مناسب برای مواجهه با این چالش است. زمینه پیشنهادی دیگر برای پردازش تصاویر سی‌تی‌اسکن، تغییر ساختار روش دومرحله‌ای از یک مدل طبقه‌بندی به همراه یک مدل قطعه‌بندی، به یک مدل بخش‌بندی به همراه یک مدل قطعه‌بندی است. این روش با شناسایی محدوده خونریزی، ناترازی پیکسلی را به‌منظور آموزش مدل قطعه‌بندی کاهش می‌دهد. 
 
\section{نتیجه‌گیری}
در این پایان‌نامه، یک روش دومرحله‌ای برای شناسایی و قطعه‌بندی خودکار خونریزی‌های درون‌جمجمه‌ای با استفاده از تصاویر سی‌تی‌اسکن ارائه شده است. استفاده از روش پس‌پردازش و تصمیم‌گیری پس از قطعه‌بندی، بهبود قابل‌توجهی در دقت و صحت نتایج داشته است. نتایج به‌دست‌آمده نشان می‌دهد که این روش می‌تواند به‌عنوان یک ابزار کارآمد در تشخیص‌های پزشکی به کار رود.
باتوجه‌به نتایج حاصل از این پژوهش، ابزار‌های هوش مصنوعی و یادگیری عمیق می‌توانند نقشی اساسی در بهبود کیفیت تشخیص‌های پزشکی داشته باشند. در آینده، می‌توان این مدل را با استفاده از مجموعه‌داده‌های متنوع‌تر و روش‌های پیچیده‌تر مانند استفاده از مکانیسم‌های توجه گسترش داد. همچنین، امکان ادغام این مدل با فرایند‌های بیمارستانی برای تشخیص زمان‌واقعی و کمک به تصمیم‌گیری پزشکان وجود دارد.
این پژوهش نشان می‌دهد که ادغام مدل‌های هوش مصنوعی با فرایند‌های پزشکی می‌تواند به بهبود کارایی و کاهش زمان و هزینه‌های تشخیص و درمان کمک کند. با توسعه بیشتر این مدل‌ها، استفاده گسترده‌تر از آن‌ها در بیمارستان‌ها و مراکز درمانی ممکن است منجر به بهبود نتایج بیماران و کاهش خطاهای پزشکی شود.
